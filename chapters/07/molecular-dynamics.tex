\chapter{Introduction to molecular dynamics}

\section{Introduction}

	\subsection{Hamilton's equations}
	The starting point to a Molecular dynamics simulation is Hamilton's equation.
	Hamilton's equation will yield Newton's equation at the end, allowing to study the system as first order differential equations.

	$$\dot{q}_\alpha = \frac{\partial\mathcal{H}}{\partial p_\alpha}\qquad\dot{p}_\alpha = - \frac{\partial\mathcal{H}}{\partial q_\alpha}$$

	The objective is to obtain a numerical result from these equation.
	Remembering that solving Hamilton's equation means integrating them keeping the energy constant.

	$$\frac{d\mathcal{H}}{dt} = \sum\limits_\alpha\biggl[\frac{\partial\mathcal{H}}{\partial q_\alpha}\dot{q}_\alpha + \frac{\partial\mathcal{H}}{\partial p_\alpha}\dot{p}_\alpha\biggr] = \sum\limits_\alpha\biggl[\frac{\partial\mathcal{H}}{\partial q_\alpha}\frac{\partial\mathcal{H}}{\partial p_\alpha}-\frac{\partial\mathcal{H}}{\partial p_\alpha}\frac{\partial\mathcal{H}}{\partial q_\alpha}\biggr] = 0$$

	All the work is done in the microcanonical ensemble.

	$$\mathcal{H}(q_1, \dots, q_{3N}, p_1, \dots, p_{3N}) = const$$

	The quantities obtained through Hamilton's equations are representative of the microcanonical ensemble.
	The time-dependent solutions will be rigorous, conformations or state that are separated by an energy barrier, local minima cannot be escaped.

	\subsection{Ergodicity}
	When a property has to be measured in an ensemble, what is measured is the average over an ensemble:

	$$A = \langle a\rangle = \frac{\int dxa(x)\delta(\mathcal{H}(x)-E)}{\int dx\delta(\mathcal{H}(x)-E)} = \lim\limits_{\tau\rightarrow\infty}\frac{1}{\tau}\int_0^\tau dta(x_t)\equiv\bar{a}$$

	Assuming that the system will visit during in dynamics all the possible state of its ensemble the average over the ensemble can be substituted over an average over time.
	Usually the time average is indicated by a bar: $\bar{a}$.
	The numerical integrator will give the value of $a(x_t)$ at different time steps and from that a discretized time average will be obtained:

	$$A = \langle a\rangle = \frac{1}{M}\sum\limits_{n=1}^M a(x_{n\Delta t})$$

	This discretized time average for sure is not the ensemble average.
	In order for this to be true the system has to have the property of ergodicity.
	Ergodicity means that in a simulation, while exploring different state and point in time, all possible states compatible with a macrostate are being explored.
	So all the state belonging to the ensemble are being explored in the phase-space.
	This is not easy to assume for complex states and depends on the energy profile of the system.


	\subsection{Basic components of a molecular dynamics simulation}
	The basic components of a MD simulation are:

	\begin{multicols}{2}
		\begin{itemize}
			\item The model: the chosen force field, the model used to represent chemical bonds and reality.
			\item Calculation of energies and forces: accurate and efficient.
				So how to compute the energies and forces, numerical steps in the model characterized by some error.
			\item The algorithm: used to integrate the equations of motion.

		\end{itemize}
	\end{multicols}

\section{Verlet algorithm}
An algorithm used to integrate the equations of motion is the Verlet algorithm.
Starting from the Taylor expansion of the coordinates at time $t+\Delta t$ and then they can be written using Newton's law.

\begin{align*}
	\vec{r}_i(t+\Delta t)&\approx \vec{r}_i(t) + \Delta t\dot{\vec{r}}_i(t) + frac{1}{2}\Delta t^2\ddot{\vec{r}}_i(t)
											 &\approx\vec{r}_i(t+\Delta t)+\Delta t\vec{v}_i(t)+\frac{\Delta t^2}{2m_i}\vec{F}_i(t) \\
\end{align*}


Similarly, integrating backward in time:

$$\vec{r}_i(t-\Delta t) \approx\vec{r}_i(t)-\Delta t\vec{v}_i(t)+\frac{\Delta t^2}{2m_i}\vec{F}_i(t)$$

Summing up the two equations the first order terms will cancel:

$$\vec{r}_i(t+\Delta t) + \vec{r}_i(t-\Delta t) = 2\vec{r}_i(t) + \frac{\Delta t^2}{m_i}\vec{f}_i(t)$$

In these equations the third order terms will also cancel, so that this sum is correct until the forth order term.
So this equation is correct up to the forth order term
So the coordinates at time $t+\Delta t$ are:

$$\vec{r}_i(t+\Delta t) = 2\vec{r}_i(t) - \vec{r}_i(t-\Delta t) + \frac{\Delta t^2}{m_i}\vec{F}_i(t)$$

In order to get the coordinates at the previous instant in time have to be stored.
The velocity can be computed using velocity's definition:

$$\vec{v}_i(t) = \frac{\vec{r}_i(t+\Delta t)-\vec{r}_i(t-\Delta t)}{2\Delta t}$$

The best thing to do is to compute the average velocity over $2$ time step as to have a numerically more stable solution.
This algorithm is time-reversible and will keep energy constant.
Some problem about it is that the velocity are the kinetic energy and is related to the temperature of the system.

	\subsection{Velocity Verlet}
	A variation over Verlet providing the same trajectory, but computing velocities and coordinates at the same time.
	Again the starting point is the Taylor expansion:

	$$\vec{r}_i(t+\Delta t) = \vec{r}_i(t) + \Delta t\vec{v}_i(t) + \frac{\Delta t^2}{2m_i}\vec{F}_i(t)$$

	Now integrating backward in time considering the velocities:

	$$\vec{r}_i(t) = \vec{r}_i*t+\Delta t) -\Delta t\vec{v}_i(t + \Delta t) + \frac{\Delta t^2}{2m_i}\vec{F}_i(t+\Delta t)$$

	By substituting $\vec{r}_i*t + \Delta t)$ with the first equation:

	$$\vec{v}_i(t+\Delta t) = \vec{v}_i(t) + \frac{\Delta t}{2m_i}[\vec{F}_i(t) + \vec{F}_i(t+\Delta t)]$$

	With the average of the forces at the two time steps.
	The velocities and the coordinates are updated at the same time.
	Time reversibility is necessary because Hamilton's equations are being solved.
	Moreover these algorithms have a symplectic structure, a property related with their numerical stability: this means that the trajectories that are obtained using this algorithms although it is not exactly the same, the errors will not diverge from the classical trajectory.

	\subsection{Initial condition}
	The coordinates of the initial condition are either taken from experimental data or guessed.
	The velocities are taken randomly from a Maxwell-Boltzmann distribution:

	$$f(v) = \biggl(\frac{m}{2\pi kT}\biggr)^\frac{1}{2}e^{-\frac{mv^2}{2kT}}$$

	Consider the Gaussian probability distribution:

	$$f(x) = \frac{1}{\sqrt{2\pi\sigma^2}}e^{-\frac{x^2}{2\sigma^2}}$$

	\subsection{Action integral}
	Consider:

	$$Q \equiv\{q_1, \dots, q_{3N}\}\qquad \dot{Q}\equiv\{\dot{q}_1, \dots, \dot{q}_{3N}\}$$

	The action integral:

	$$A[Q] = \int_{t_1}^{t_2}\mathcal{L}(Q(t), \dot{Q}(t))dt$$

	The path $Q$ that renders the action stationary is:

	\begin{multicols}{4}
		\begin{itemize}
			\item $Q(t_1) = Q_1$.
			\item $Q(t_2) = Q_2$.
			\item $\dot{Q}(t_1) = \dot{Q}_1$.
			\item $\dot{Q}(t_2) = \dot{Q}_2$.
		\end{itemize}
	\end{multicols}

	Now:

	\begin{multicols}{2}
		\begin{itemize}
			\item $\delta Q(t_1) = \delta Q(t_2) = 0$.
			\item $\delta\dot{Q}(t_1) = \delta\dot{Q}(t_2) = 0$.
		\end{itemize}
	\end{multicols}

	\begin{align*}
		\delta A &= \int_{t_1}^{t_2}\mathcal{L}(Q(t) + \delta Q(t), \dot{Q}(t)+\delta\dot{Q}(t))dt - \int_{t_1}^{t_2}\mathcal{L}(Q(t), \dot(Q)(t))dt = \\
						 &=\int_{t_1}^{t_2}\sum\limits_{\alpha=1}^{3N}\biggl[\frac{\partial\mathcal{L}}{\partial q_\alpha}\delta q_\alpha(t) + \frac{\partial\mathcal{L}}{\partial\dot{q}_\alpha}\delta\dot{q}_\alpha(t)\biggr]dt=\\
						 &=\sum\limits_{\alpha=1}^{3N}\frac{\partial\mathcal{L}}{\partial\dot{q}_\alpha}\delta q_\alpha(t)|_{t_1}^{t_2} + \int_{t_1}^{t_2}\sum\limits_{\alpha=1}^{3N}\biggl[\frac{\partial\mathcal{L}}{\partial q_\alpha}\delta q_\alpha(t) - \frac{d}{dt}\biggl(\frac{\partial\mathcal{L}}{\partial\dot{q}_\alpha}\biggr)\delta q_\alpha(t)\biggr] dt = 0
	\end{align*}

	Thus:

	$$\frac{\partial\mathcal{L}}{\partial q_\alpha} - \frac{d}{dt}\biggl(\frac{\partial\mathcal{L}}{\partial\dot{q}_\alpha}\biggr) = 0\Rightarrow \frac{d}{dt}\biggl(\frac{\partial\mathcal{L}}{\partial\dot{q}_\alpha}\biggr)-\frac{\partial\mathcal{L}}{\partial q_\alpha} = 0$$

\section{Constraints}

\begin{itemize}
	\item Holonomic constraints: $\sigma_k(q_1, \dots, q_{3N}, t) = 0\qquad k = 1, \dots, N_C$.
	\item Nonholonomic constraints: $\zeta(q_1, \dots, q_{3N}, \dot{q}_1, \dots, \dot{q}_{3N}) = 0$.
\end{itemize}

For example consider:

$$\frac{1}{2}\sum\limits_{i}m_i\dot{\vec{r}}_i^2-C = 0$$

Considering the minimal set of coordinates in generalized coordinates $3N-N_C$.
This is an example of spherical coordinates at fixed radius.

	\subsection{Differential forms}

	$$\sum\limits_{\alpha = 1}^{3N} a_{k\alpha}dq_\alpha + a_{kt}dt = 0\qquad k = 1, \dots, N_C$$


		\subsubsection{Holomonic constraints}

		$$\sum\limits_{\alpha=1}^{3N}\frac{\partial\sigma_k}{\partial q_\alpha}dq_\alpha + \frac{\partial\sigma_k}{\partial t} dt = 0\qquad k = 1, \dots, N_N\qquad a_{k\alpha} = \frac{\partial\sigma_k}{\partial q_\alpha}\qquad a _{kt} = \frac{\partial\sigma_k}{\partial t}$$

		\subsubsection{Nonholonomic constraints}

		$$\frac{1}{2}\sum\limits_i m_i\dot{\vec{r}}_i^2 - C = 0\Rightarrow\frac{1}{2}\sum\limits_i m_i\dot{\vec{r}}_i\frac{d\vec{r}_i}{dt} - C = 0\Rightarrow\frac{1}{2}\sum\limits_i m_i\dot{\vec{r}}_id\vec{r}_i-Cdt = 0$$

		$$a_{1i}=\frac{1}{2}m_i\dot{\vec{r}}_i\qquad a_{1t} = -C$$

		So the integrable form:

		$$\sum\limits_{\alpha=1}^{3N}a_{k\alpha}dq_\alpha=0$$

			\paragraph{Lagrange multipliers}

			$$\int_{t_1}^{t_2}\sum\limits_{\alpha=1}^{3N}\biggl[\frac{\partial\mathcal{L}}{\partial q_\alpha} - \frac{d}{dt}\biggl(\frac{\partial\mathcal{L}}{\partial\dot{q}_\alpha}\biggr) + \sum\limits_{k=1}^{N_N}\lambda_ka_{k\alpha}\biggr]\delta q_\alpha(t)dt = 0$$

			$$\frac{d}{dt}\biggl(\frac{\partial\mathcal{L}}{\partial\dot{q}_\alpha}\biggr) - \frac{\partial\mathcal{L}}{\partial q_\alpha} = \sum\limits_{k=1}^{N_C}\lambda_k a_{k\alpha}$$

			$$\sum\limits_{\alpha=1}^{3N}a_{k\alpha}\dot{q}_\alpha + a_{kt} = 0\qquad k = 1, \dots, N_C$$

			So there are $3N+N_C$ equations for $3N+N_c$ unknowns.

	\subsection{Hamiltonian formulation}
	Considering the time-independent holonomic constraints:

	$$\begin{cases}\dot{q}_\alpha = \frac{\partial\mathcal{H}}{\partial p_\alpha}\\\dot{q}_\alpha = -\frac{\partial\mathcal{H}}{\partial q_\alpha} - \sum\limits_{k=1}^{N_C}\lambda_ka_{k\alpha}\\\sum\limits_{\alpha=1}^{3N}a_{k\alpha}\frac{\partial\mathcal{H}}{\partial p_\alpha} = 0\end{cases}$$

	So that:

	\begin{align*}
		\frac{d\mathcal{H}}{dt} &= \sum\limits_\alpha\biggl[\frac{\partial\mathcal{H}}{\partial q_\alpha}\dot{q}_\alpha +\frac{\partial\mathcal{H}}{\partial p_\alpha}\biggr] = \\
														&=\sum\limits_\alpha\biggl[\frac{\partial\mathcal{H}}{\partial q_\alpha}\frac{\partial\mathcal{H}}{\partial p_\alpha} + \frac{\partial\mathcal{H}}{\partial p_\alpha}\biggl(\frac{\partial\mathcal{H}}{\partial q_\alpha} + \sum\limits_k \lambda_ka_{k\alpha}\biggr)\biggr] = \\
														&=\sum\limits_k\lambda_k\sum\limits_\alpha\frac{\partial\mathcal{H}}{\partial p_\alpha}a_{k\alpha} = 0
	\end{align*}

	So no work is done on a system by the imposition of holonomic constraints.

	\subsection{Constraints in a simulation}

	$$m_i\ddot{r}_i = \vec{F}_i + \sum\limits_{k=1}^{N_C}\lambda_k\nabla_i\sigma_k$$

	$$\frac{d}{dt}\sigma_k(\vec{r}_1, \dots, \vec{r}_N) = 0\Rightarrow\dot{\sigma}_k = \sum\limits_{i=1}^N\nabla_i\sigma_k\cdot\dot{\vec{r}}_i = 0$$

	Including the constraints on the integration algorithm:

	$$\vec{r}_i(\Delta t) = \vec{r}_i(0) + \Delta t\vec{v}_i(0) + \frac{\Delta t^2}{2m_i}\vec{F}_i(0) + \frac{\Delta t^2}{2m_i}\sum\limits_{k}\lambda_k\nabla_i\sigma_k(0)$$

	TO obtain $\lambda_k$:

	$$\vec{r}_i' = \vec{r}_i(0) + \Delta t\vec{v}_i(0) + \frac{\Delta t^2}{2m_i}\vec{F}_i(0)$$

	$$\vec{r}_i(\Delta t) = \vec{r}'_i + \frac{1}{m_i}\sum\limits_k\tilde{\lambda}_k\nabla_i\sigma_k(0)$$

	So that:

	$$\tilde{\lambda}_k = \frac{\Delta t^2}{2}\lambda_k$$

	\subsection{Constraint condition}

	$$\sigma_l)\vec{r}_1(\Delta t), \dots, \vec{r}_N(\Delta t)) = 0\qquad l = 1, \dots, N_C$$

	$$\sigma_l\biggl(\vec{r}_1' + \frac{1}{m_1}\sum\limits_k\tilde{\lambda}_k\nabla_1\sigma_k(0), \dots, \vec{r}_N' + \frac{1}{m_N}\sum\limits_k\tilde{\lambda}_k\nabla_N\sigma_k(0)\biggr) = 0\qquad l = 1, \dots, N_C$$

	Considering SHAKE, an iterative solution from an initial guess $\tilde{\lambda}_k^{(1)}$:

	$$\vec{r}_i^{(1)} = \vec{r}_i' + \frac{1}{m_i}\sum\limits_k\tilde{\lambda}_k^{(1)}\nabla_1\sigma_k(0)$$

	The exact solution: $\tilde{\lambda}_k = \tilde{\lambda}_k^{(1)} + \delta\tilde{\lambda}_k^{(1)}$:

	$$\vec{r}_i(\Delta t) = \vec{r}_i^{(1)} + \frac{1}{m_i}\sum\limits_k\delta\tilde{\lambda}_k^{(1)}\nabla_1\sigma_k(0)$$

	$$\sigma_l\biggl(\vec{r}_1^{(1)} = \frac{1}{m_1}\sum\limits_k\delta\tilde{\lambda}_k\nabla_1\sigma_k(0), \dots, \vec{r}_N^{(1)} + \frac{1}{m_N}\sum\limits_{k}\delta\tilde{\lambda}_k\nabla_N\sigma_k(0)\biggr) = 0$$

	Considering the Taylor series:

	$$\sigma_l(\vec{r}_1^{(1)}, \dots, \vec{r}_N^{(1)}) + \sum\limits_{i=1}^N\sum\limits_{k=1}^{N_C}\frac{1}{m_i}\nabla_i\sigma_l(\vec{r}_1^{(1)}, \dots, \vec{r}_N^{(1)})\nabla_i\sigma_k(\vec{r}_1(0), \dots, \vec{r}_N(0))\delta\tilde{\lambda}_k\approx 0$$

\section{Possible algorithms}

$$\sigma_l(\vec{r}_1^{(1)}, \dots, \vec{r}_N^{(1)}) + \sum\limits_{i=1}^N\sum\limits_{k=1}^{N_C}\frac{1}{m_i}\nabla_i\sigma_l(\vec{r}_1^{(1)}, \dots, \vec{r}_N^{(1)})\nabla_i\sigma_k(\vec{r}_1(0), \dots, \vec{r}_N(0))\delta\tilde{\lambda}_k\approx 0$$

\begin{itemize}
	\item Direct inversion: Matrix-shake or M-SHAKE: hte procedure must be repeated because the equation above is a linear approximation.
	\item Quick trick: only diagonal element are considered without any matrix inversion.
	\item The same procedure can be employed for velocities like in RATTLE.
	\item LINCS or linear constraint solver is based on the same principle and implemented in GROMACS.
\end{itemize}
