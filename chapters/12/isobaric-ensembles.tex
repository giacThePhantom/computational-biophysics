\graphicspath{{chapters/12/images/}}
\chapter{The isobaric ensembles}

\section{Introduction}
Why do we need to study the isobaric ensembles? The reason is that in MD we investigate systems of biological relevance, therefor all simulation should be run on constant temperature and pressure (atmospheric pressure).  
	\subsection{Legendre transform of E}
	In order to get to that point, we first start by performing a Legendre transform of the energy (isobaric-isoentalpic ensemble). 
We have already saw it when introduced canonical from microcanonical : legendre transform of the enrgy wrt to its derivative wrt to its entropy. We wanted to express the legendre tran of the energy in terms of the temperature, something that was exactly its derivate wrt to its entropy.
We are introducing pressure, derv of the enrgy wrt to volume (-). We want to study this new energy, this will the the function of the number of molecules. Vol will be expressed in terms of the pressure.

	$$\tilde{f}(s) = f(x(s))-sx(s)\qquad s = f'(x)$$

	$$\tilde{E}(N, \frac{\partial E}{\partial V}, S) = E(N, V(P), S) - \biggl(\frac{\partial E}{\partial V}\biggr)_{N, S}V(N, \frac{\partial E}{\partial V}, S)$$
	
	

	Enthalpy:

	$$H(N, P, S) = E(N, P, S) + PV(N, P, S)$$
	
	The quantity $H$ is also called enthalpy. As it is usually done for thermodynamic potentials, we derive the expression for its differential form, that i equal to:
	
	$$dH = dE + PdV + VdP = TdS - PdV + \mu dN + PdV + VdP$$
	
	We know from the first principal of thermodynamics that $dE$ is equal to $TdS - PdV + \mu dN$, where $\mu$ is the chemical potential. Then we are left with the following expression: 

	$$dH = TdS + \mu dN + VdP$$

From this we can derive the usually derivatives:

	$$T = \biggl(\frac{\partial H}{\partial S}\biggr)_{N, P} \qquad \langle V\rangle = \biggl(\frac{\partial H}{\partial P}\biggr)_{N, S} \qquad \mu = \biggl(\frac{\partial H}{\partial N}\biggr)_{P, S}$$
	
	The volume is averaged because in this ensemble it is not fixed, but will vary. 

	\subsection{Legendre transform of A}
	 In the exact same way we can also apply the Legendre transform to the Helmholtz free energy, with the formulas being very similar to before: 

	$$\tilde{f}(s) = f(x(s))-sx(s)\qquad s = f'(x)$$
	
	Again it is a Legendre transform wrt its derivative wrt volume, so the derivative of A wrt to volume. We know that derivative is related to the pressure, being actually - pressure, so the LT will be equal to the Helmholtz itself, minus the derivative of $A$ wrt volume times volume, which will be expressed in terms of the derivative of $A$ wrt to volume. 

	$$\tilde{A}(N, \frac{\partial A}{\partial V}, T) = A(N, V(P), T)- \biggl(\frac{\partial A}{\partial V}\biggr)_{N, T}V(N, \frac{\partial A}{\partial V}, T)$$
	
	We derived a thermodynamic potential, that we call Gibbs free energy:

	$$G(N, P, T) = A(N, P, T) + PV(N, P, T)$$

	The Gibbs free energy is the Legendre transform of the $A$, like the enthalpy is the Legendre transform of the energy. 
	Let's now take the infinitesimal variation of $G$: 
	$$dG = dA + PdV + VdP = -PdV + \mu dN - SdT + PdV + VdP$$
	
	Like before, the $PdV$ cancel out and we are left with the following expression:

	$$dG = \mu dN + VdP - SdT$$
	
	We can now tale the thermodynamic derivatives:

	$$S = -\biggl(\frac{\partial G}{\partial T}\biggr)_{N, P}\qquad\langle V\rangle = \biggl(\frac{\partial G}{\partial P}\biggr)_{N, T}\qquad \mu = \biggl(\frac{\partial G}{\partial N}\biggr)_{P, T}$$
	
	We obtained:
	\begin{itemize}
	\item The isobaric-isoenthalpic ensemble as a Legendre transform of the microcanonical ensemble;
	\item And we have obtained the isobaric-isothermal ensemble as the legendre trasnform of the canical ensemble.
	\end{itemize}
	
	This is also the same process when we go from one simulation to the other (note that MD simulations are in fact carried in the NPT ensemble).

	\subsection{Phase space distribution of the isoenthalpic-isobaric ensemble}
	We are now keeping the enthalpy and the pressure fixed in mmuch the same way in which we keep the enrgy and the volume fixed in the microcanonical ensemble. 
	The conserved quantity is therefore $H$, the enthalpy:

	$$H = \mathcal{H}(v) + PV$$
	
	The phase-space  distribution must be the solution for the Liouville equation and we know that the solution to the Liouville equation is some function of the hamiltonian, but since we have to keep this quantity fixed and this quantity is itself a function of the hamiltonian, it results in the solution being the delta function of the quantity below:

	$$f(x) = F(\mathcal{H}(x)) = \mathcal{M}\delta(\mathcal{H}(x)+PV-H)$$

	We now write down a function that is the analogous of configurations that we had in the case of the microcanonical ensemble. Similar to the $\Omega$ we had for the microcanonical ensamble, but now the pressure varyes, it does so by changing from $0$ to infinity (in priniciple). We need therefore to also integrate over the volume. 
	
	Integrating over the constant enthalpy hypersurface:

	$$\Gamma(N, P, H) = \mathcal{M}\int_0^{\infty}dV\int d^N\vec{p}\int_{\mathcal{D}(V)}d^N\vec{r}\delta(\mathcal{H}(\vec{r}, \vec{p}) + PV-H)$$
	
	The integral $int_0^{\infty}dV$ is ont eh left side of hte equation beacause the molecules (coordinates) are integrated over the volume, dependind on it.
	
	This is very similar to the microcanonical ensemble, when we correlated the function $\Omega$ to the entropy. We can now do something similar, and derive analogous Boltzmann relation. The entropy, as a function of the number of particles, the presssure and the enthalpy, is
	
	$$S(N, P, H) = k \, ln \Gamma(N, P, H)$$

	$$dH = TdS + \mu dN + VdP$$
	
	We can therefore derive all the thermodynamic derivates to describe the system:
 
	$$\frac{1}{T} = \biggl(\frac{\partial S}{\partial H}\biggr)_{N, P}\qquad \frac{\langle V\rangle}{T} = -\biggl(\frac{\partial S}{\partial P}\biggr)_{N, H}\qquad \frac{\mu}{T} = -\biggl(\frac{\partial S}{\partial N}\biggr)_{P, H}$$

\section{Isothermal-isobaric ensemble}
Let's now study the isothermal-isobaric ensemble. In order to do this, let's write down the thermodynamic variables, ass we have seen before for the microcanical.
Note that the reasoning behind this derivations is the same of the canonical and microcanonical ensemble! 
The microcanonical and the isothermal-isobaric: what we will do now is to go from the analogous microcanonical (isobaric-isoenthalpic) to the analogous canonical (isothermal-isobaric). 
Figure \ref{fig:isobar} also proves the similarity between the system, with the only difference being the piston at the bottom, since the pressure is not constant (in both compartment).

\begin{figure}
\center
\includegraphics[scale=0.4]{isobar}
\label{fig:isobar}
\caption{Two systems in contact with a common thermal reservoir at temperature T. System
$1$ has $N_1$ particles in a volume $V_1$ ; system $2$ has $N_2$ particles in a volume $V_2$. Both $V_1$ and $V_2$ can vary.}
\end{figure}

\begin{multicols}{2}
	\begin{itemize}
		\item $E = E_1 + E_2\quad E_2\gg E_1$.
		\item $N = N_1 + N_2\quad N_2\gg N_1$.
		\item $V = V_1 + V_2\quad V_2\gg V_1$
		\item $\mathcal{H}(x) = \mathcal{H}_1(x_1) + \mathcal{H}_2(x_2)$.
	\end{itemize}
\end{multicols}


At fixed volumes $V_1$ and $V_2$:

$$Q(N, V, T) = C_N\int dx_1dx_2 e^{-\beta[\mathcal{H}_1(x_1) + \mathcal{H}_2(x_2)]} = g(N, N_1, N_2)C_{N_1}\int dx_1 e^{-\beta\mathcal{H}_1(x_1)}C_{N_2}\int dx_2 e^{-\beta\mathcal{H}_2(x_2)}$$

Which is the same expression for the micro canonical ensemble. $C_{N_1}$ and $C_{N_2}$ are the two constants that we would need to separate the system in two canonical ensemble with the same temperature $\beta$. 

The partition function for the system is: 

$$Q(N, V, T) \propto Q(N_1, V_1, T)Q(N_2, V_2, T)$$

But what happens if the volume is free to vary?

	\subsection{Phase space distribution}
	Combined system:

	$$f(x) = \frac{C_Ne^{-\beta\mathcal{H}(x)}}{Q(N, V, T)} = \frac{g(N, N_1, N_2)}{Q(N, V, T)}C_{N_1}e^{-\beta\mathcal{H}_1(x_1)}C_{N_2}e^{-\beta\mathcal{H}_2(x_2)}$$

We have to "wash out" the variables that describe the system $2$. So we need to integrate over the $x_2$ variables. 

	$$f(x_1) = \frac{g(N, N_1, N_2)}{Q(N, V, T)}C_{N_1}e^{-\beta\mathcal{H}_1(x_1)}C_{N_2}\int dx_2 e^{-\beta\mathcal{H}_2(x_2)} = \frac{Q_2(N_2, V-V_1, T)}{Q(N, V, T)}g(N, N_1, N_2)C_{N_1}e^{-\beta\mathcal{H}_1(x_1)}$$
	
	The quantity $C_{N_2}\int dx_2 e^{-\beta\mathcal{H}_2(x_2)}$ is exactly the partition function for the system $2$, considered as a canonical ensemble, with a fixed number of particles and a given volume and temperature. Thus being equal to $Q_2$: let's focus on this quantity. 

	First, the normalization is correct: $\int dV_1\int dx_1f(x_1, V_1) = 1$.
	
	Generally, we know that the partition function ($Q_2$ in this case) is equal to $e^{\beta[A]}$. So let's write this equation with the correct argument for the Helmholtz free energy. 

	$$\frac{Q_2(N_2, V-V_1, T)}{Q(N, V, T)} = e^{-\beta[A(N-N_1, V-V_1, T) - A(N, V, T)]}$$

Through a Taylor expansion:

	$$A(N-N_1, V-V_1, T) = A(N, V, T)-N_1\frac{\partial A}{\partial N}|_{N_1 = 0, V_1 = 0} = A(N, V, T)-\mu N_1 + PV_1$$
	
	The phase-space distribution for variable $x_1$ is:

	$$f(x_1) = g(N, N_1, N-N_1) e^{\beta\mu N_1}e^{-\beta P V_1}C_{N_1}e^{-\beta\mathcal{H}_1(x_1)}\qquad I_{N_1} = \frac{1}{V_0N_1!h^{3N_1}}$$

	We define a quantity $\Delta(N, P, T)$ to be the integral of hte entire phase-space, including the volume, of the constant $I_N$. The phase-space distribution is simply $e^{-\beta(\mathcal{H}(x) + PV)}$, so basically the analogous distribution function of the canonical ensemble, but instead of having the energy in the Boltzmann factor we have the enthalpy. 
	
	$$\Delta(N, P, T) = I_N\int_0^{\infty}dV\int dxe^{-\beta(\mathcal{H}(x) + PV)}\Rightarrow e^{\beta\mu N}\Delta(N, P, T) = 1$$
	
	Because of Euler (next chapter), the quantity $\mu N$ is exactly $G$:

	$$\Delta(N, P, T) = e^{-\beta\mu N} = e^{-\beta G(N, P, T)}$$
	
	Now we have an analogous relation: $e^{\beta A} \rightarrow e^{\beta G}$. 
	
	$$\Delta(N, P, T) = \frac{1}{V_0N!h^{3N}}\int_0^{\infty}dVe^{-\beta PV}\int dxe^{-\beta\mathcal{H}(x)} = \frac{1}{V_0}\int_0^{\infty}dVe^{-\beta PV}Q(N, V, T)$$
	
	\subsubsection{Further proof}
	A further proof that this is indeed a "good" system is provided, starting this time from the Gibbs free energy.

	$$G = A + P\langle V \rangle = \langle E + PV\rangle - TS = \langle\mathcal{H}(x) + PV\rangle + T\frac{\partial G}{\partial T}$$

We can obtain the average $langle E + PV\rangle$ by performing the average over the phase-space distribution we just introduced. We need to integrate over the volume and all the coordinates. The quantity $(\mathcal{H}(x) + PV)$ (remember that $V$ now is a constant) must be weighted by the corresponding Boltzmann factor, which includes not only the hamiltonian but also $PV$.

	$$\langle E + PV\rangle = \frac{I_N\int_0^{\infty} dV\int dx(\mathcal{H}(x) + PV)e^{-\beta(\mathcal{H}(x)+PV)}}{I_N\int_0^{\infty}dV\int dxe^{-\beta(\mathcal{H}(x) + PV)}} = -\frac{1}{\Delta(N, P, T)}\frac{\partial \Delta(N, P, T)}{d\beta} = -\frac{\partial\ln\Delta(N, P, T)}{\partial \beta}$$
	

	$$G(N, P, \beta) = -\frac{\partial\ln\Delta(N, P, \beta)}{\partial\beta}-\beta\frac{\partial G}{\partial \beta}$$

	If we plug in the following solution: $G(N, P, \beta) = -\frac{1}{\beta}\ln\Delta(N, P, \beta)$ in the previous differential equation we get $0$.
	
	We now have all the means to describe the system. 

	$$\langle V\rangle = \biggl(\frac{\partial G}{\partial P}\biggr)_{N, T}\qquad S = -\biggl(\frac{\partial G}{\partial T}\biggr)_{N, P}$$

	\subsection{Maxwell's square}
	
	Let's put everything into perspective. 

	We have seen in the NPT enseble that we can obtain the average volume as the derivative wrt pressure of the Gibbs free energy, the entropy can be taken as the derivative wrt temperature (keeping fixed N and T).
	\begin{multicols}{2}
		\begin{itemize}
			\item $\langle V\rangle = \biggl(\frac{\partial G}{\partial P}\biggr)_{N, T}$.
			\item $S = -\biggl(\frac{\partial G}{\partial T}\biggr)_{N, P}$.
		\end{itemize}
	\end{multicols}
	
	We also found the following equation from the isobaric-isoenthalpic ensemble:

	\begin{multicols}{2}
		\begin{itemize}
			\item $T = \biggl(\frac{\partial H}{\partial S}\biggr)_{N, P}$.
			\item $\langle V\rangle = \biggl(\frac{\partial H}{\partial P}\biggr)_{N, S}$.
		\end{itemize}
	\end{multicols}

The following are coming from the microcanonical ensemble :
	\begin{multicols}{2}
		\begin{itemize}
			\item $T = =\biggl(\frac{\partial U}{\partial S}\biggr)_{N, V}$.
			\item $P = - \biggl(\frac{\partial U}{\partial V}\biggr)_{N, S}$.
		\end{itemize}
	\end{multicols}
	
And the canonical ensemble:
	\begin{multicols}{2}
		\begin{itemize}
			\item $P = -\biggl(\frac{\partial A}{\partial V}\biggr)_{N, T}$.
			\item $S = - \biggl(\frac{\partial A}{\partial T}\biggr)_{N, V}$.
		\end{itemize}
	\end{multicols}

	\begin{figure}[H]
		\includegraphics[scale = 0.1]{maxwell_square}
		\centering
		\caption{Maxwell's square}
	\end{figure}

	\subsection{Pressure viral theorem}

	$$P^{(int)} =\langle\mathcal{P}(\vec{r}, \vec{p})\rangle = \biggl\langle\frac{1}{3V}\sum\limits_i\biggl[\frac{\vec{p}_i^2}{m_i} + \vec{F}_i\cdot\vec{r}_i\biggr]\biggr\rangle = kT\frac{\partial\ln Q}{\partial V}$$

	\begin{align*}
		\langle P^{(int)}\rangle &= \frac{1}{\Delta(N, P, T)}\int_0^{\infty}dVe^{-\beta PV}Q(N, V, T)\frac{kT}{Q}\frac{\partial Q}{\partial V} = \frac{kT}{\Delta(N, P, T)}\int_0^{\infty}dVe^{-\beta PV}\frac{\partial Q}{\partial V}=\\
														 &= \frac{kT}{\Delta(N, P, T)}e^{-\beta PV}Q(N, V, T)|_0^{\infty}-\frac{kT}{\Delta(N, P, T})\int_0^{\infty}dV\biggl(-\frac{P}{kT}\biggr)e^{-\beta PV}Q(N, V, T) = \\
														 &=\frac{P}{\Delta(N, P, T)}\int_0^{\infty}dVe^{-\beta PV}Q(N, V, T) = P
	\end{align*}

	The volume-averaged internal pressure is equal to the external pressure.

	\subsection{Work virial theorem}

	$$P^{(int)}V = kTV\frac{\partial \ln Q}{\partial V}$$

	\begin{align*}
		\langle P^{(int)}V\rangle &= \frac{1}{\Delta(N, P, T)}\int_0^{\infty}dVe^{-\beta PV}Q(N, V, T)\frac{kTV}{Q}\frac{\partial Q}{\partial V} = \frac{kT}{\Delta(N, P, T)}\int_0^{\infty}dVe^{-\beta PV}V\frac{\partial Q}{\partial V} = \\
															&=\frac{kT}{\Delta(N, P, T)}e^{-\beta PV}VQ(N, V, T)|_{0}^{\infty}-\frac{kT}{\Delta(N, P, T)}\int_0^{\infty}dV\frac{\partial}{\partial V}(Ve^{-\beta PV})Q(N, V, T)=\\
															&= \frac{1}{\Delta(N, P, T)}\biggl[-kT\int_0^{\infty}dVe^{-\beta PV}Q(N, V, T) + P\int_0^{\infty}dVVe^{-\beta PV}Q(N, V, T)\biggr]=\\
															&=-kT+P\langle V\rangle \Rightarrow \langle P^{(int)}V\rangle + kT = P\langle V\rangle
	\end{align*}

	There is an extra degree of freedom.

\section{Andersen's Hamiltonian}

$$\mathcal{H}_A = \sum\limits_{i=1}^N\frac{V^{-\frac{2}{3}}\pi_i^2}{2m_i}+ U(V^\frac{1}{3}\vec{s}_1, \dots, V^{\frac{1}{3}}\vec{s}_N) + \frac{p_V^2}{2W} + PV\qquad W = (3N+1)kT\tau_b^2$$

Hamilton's equations:

\begin{itemize}
	\item $\dot{\vec{s}}_i = \frac{\partial \mathcal{H}_A}{\partial\pi_i} = \frac{V^{-\frac{2}{3}}\pi_i}{m_i}$.
	\item $\dot{\pi}_i = -\frac{\partial\mathcal{H}_A}{\partial\vec{s}_i} = -\frac{\partial U}{\partial (V^{\frac{1}{3}}\vec{s}_i)}V^{\frac{1}{3}}$.
	\item $\dot{V} = \frac{\partial\mathcal{H}_A}{\partial p_V}-\frac{p_V}{W}$.
	\item $\dot{p}_V = -\frac{\partial\mathcal{H}_A}{\partial V} = \frac{1}{3}V^{-\frac{5}{3}}\sum\limits_{i=1}^N\frac{\pi_i^2}{m_i}-\frac{1}{3}V^{-\frac{2}{3}}\sum\limits_{i=1}^N\frac{\partial U}{\partial(V^{\frac{1}{3}}\vec{s}_i)}\cdot\vec{s}_i-P$.
\end{itemize}

Inverting the transformation

$$s_i = V^{-\frac{1}{3}}\vec{r}_i\Rightarrow \dot{\vec{s}}_i = V^{-\frac{1}{3}}\dot{\vec{r}}_i-\frac{1}{3}V^{-\frac{4}{3}}\dot{V}\vec{r}_i$$

$$\pi_i = V^{\frac{1}{3}}\vec{p}_i \Rightarrow \dot{\pi}_i = V^{\frac{1}{3}}\dot{\vec{p}}_i + \frac{1}{3}V^{-\frac{2}{3}}\dot{V}\vec{p}_i$$

\begin{itemize}
	\item $\dot{\vec{s}}_i = \frac{\partial \mathcal{H}_A}{\partial\pi_i} = \frac{V^{-\frac{2}{3}}\pi_i}{m_i}$.
	\item $\dot{\pi}_i = -\frac{\partial\mathcal{H}_A}{\partial\vec{s}_i} = -\frac{\partial U}{\partial (V^{\frac{1}{3}}\vec{s}_i)}V^{\frac{1}{3}}$.
	\item $\dot{V} = \frac{\partial\mathcal{H}_A}{\partial p_V}-\frac{p_V}{W}$.
	\item $\dot{p}_V = -\frac{\partial\mathcal{H}_A}{\partial V} = \frac{1}{3}V^{-\frac{5}{3}}\sum\limits_{i=1}^N\frac{\pi_i^2}{m_i}-\frac{1}{3}V^{-\frac{2}{3}}\sum\limits_{i=1}^N\frac{\partial U}{\partial(V^{\frac{1}{3}}\vec{s}_i)}\cdot\vec{s}_i-P$.
\end{itemize}

Incompressible.

	\subsection{Andersen's equations}

	\begin{multicols}{2}
		\begin{itemize}
			\item $\dot{\vec{r}}_i = \frac{\vec{p}}{m_i} + \frac{\dot{V}}{3V}\vec{r}_i\Rightarrow\dot{\vec{r}}_i = \frac{\vec{p}_i}{m_i} + \frac{\dot{V}}{3V}\vec{r}_i$
			\item $\dot{\vec{p}}_i = -\frac{\partial U}{\partial\vec{r}_i} -\frac{\dot{V}}{3V}\vec{p}_i\Rightarrow \dot{\vec{p}}_i = -\frac{\partial U}{\partial\vec{r}_i}-\frac{\dot{V}}{3V}\vec{p}_i$.
			\item $\dot{V} = \frac{p_V}{W}\Rightarrow\dot{V} = \frac{p_V}{W}$.
			\item $\dot{p}_V = \frac{1}{3V}\sum\limits_{i=1}^N\biggl[\frac{\vec{p}_i^2}{m_i}-\frac{\partial U}{\partial\vec{r}_i}\cdot\vec{r}_i\biggr]-P\Rightarrow\dot{p}_V = \frac{1}{3V}\sum\limits_{i=1}^N\biggl[\frac{\vec{p}_i^2}{m_i}-\frac{\partial U}{\partial\vec{r}_i}\cdot\vec{r}_i\biggr]-P$.
		\end{itemize}
	\end{multicols}

	The conserved quantity:

	$$\mathcal{H}' = \sum\limits_{i=1}^N\frac{\vec{p}_i^2}{2m_i} + U(\vec{r}_1, \dots, \vec{r}_N) + \frac{p_V^2}{2W}+PV$$

	The partition function:

	$$\Omega_P = \int dp_V\int_0^{\infty}\int d^N\vec{p}\int_{\mathcal{D}(V)}d^N\vec{r}\delta\biggl(\\mathcal{H}(\vec{r},\vec{p}) + \frac{p_V^2}{2W}+PV-H\biggr)$$

	Virial theorem:

	$$\biggl\langle\frac{p_V^2}{2W}\biggr\rangle = k\frac{T}{2}\Rightarrow \mathcal{H}(\vec{r},\vec{p}) + PV\text{ is conserved}$$

\section{MTK algorithm (NPT)}

\begin{multicols}{2}
	\begin{itemize}
		\item $\dot{\vec{r}}_i = \frac{\vec{p}_i}{m_i} + \frac{p_\epsilon}{W}\vec{r}_i$.
		\item $\dot{\vec{p}}_i = -\frac{\partial U}{\partial\vec{r}_i} - \frac{p_\epsilon}{W}\vec{p}_i$.
		\item $\dot{V} = \frac{dVp_\epsilon}{W}$.
		\item $\dot{p}_\epsilon = dV(\mathcal{P}^{(int)}-P)$.
	\end{itemize}
\end{multicols}

$$\epsilon = \frac{1}{3}\ln\frac{V}{V_0}\Rightarrow\dot{\epsilon} = \frac{\dot{V}}{3V}=\frac{p_\epsilon}{W}$$

Compressibility:

\begin{align*}
	\kappa & = \sum\limits_{i=1}^N\biggl[\frac{\partial}{\partial\vec{r}_i}\cdot\dot{\vec{r}}_i + \frac{\partial}{\partial\vec{p}_i}\cdot\dot{\vec{p}}_i\biggr] + \frac{\partial\dot{V}}{\partial V} + \frac{\partial\dot{p}_V}{\partial p_V} = \\
				 &= dN\frac{p_\epsilon}{W}-dN\frac{p_\epsilon}{W} = d\frac{p_\epsilon}{E} = \frac{\dot{V}}{V}
\end{align*}

To obtain incompressible equations that conserve $\mathcal{H}(\vec{r},\vec{p}) + PV + \frac{p_V^2}{2W}$:

$$\dot{\vec{p}}_i = -\frac{\partial U}{\partial\vec{r}_i} - \biggl(1+\frac{d}{N_F}\biggr)\frac{p_\epsilon}{W}\vec{p}_i\qquad \dot{p}_\epsilon = dV(\mathcal{P}^{(int)}-P) + \frac{d}{N_f}\sum\limits_{i=1}^N\frac{\vec{p}_i^2}{m_i}$$

	\subsection{Langevin piston}

	\begin{itemize}
		\item $\dot{\vec{r}}_i = \frac{\vec{p}_i}{m_i} + \frac{\dot{V}}{3V}\vec{r}_i$.
		\item $\dot{\vec{p}}_i = -\frac{\partial U}{\partial\vec{r}_i}-\frac{\dot{V}}{3V}\vec{p}_i$.
		\item $\dot{V} = \frac{p_V}{W}$.
		\item $\dot{p}_V = \frac{1}{3V}\sum\limits_{i=1}^N\biggl[\frac{\vec{p}_i^2}{m_i}-\frac{\partial U}{\partial\vec{r}_i}\cdot\vec{r}_i\biggr]-P-\gamma\dot{V}+R(t)$.
	\end{itemize}

	$$\langle R(0)R(t)\rangle = \frac{2\gamma kT}{W}\delta(t)$$
