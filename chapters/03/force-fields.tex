\chapter{Semi-empirical force fields}

\section{Potential energy surface}
The potential energy surface or PES is the landscape of what values the potential energy of an atom can assume.
Different points can be recognized like:

\begin{multicols}{2}
	\begin{itemize}
		\item Saddle point.
		\item Local maximums.
		\item Local minimum.
	\end{itemize}
\end{multicols}

	\subsection{Bond stretching}
	Let $U(r)$ the function of the potential energy, now the PES can be determined by vibrational spectroscopy.
	Let $r_{eq}$ the distance for which the potential energy is minimal, then performing a Taylor expansion of the potential energy function:

	\begin{align*}
		U(r)&=U(r_{eq}) + \frac{dU}{dr}|_{r=r_{eq}}(r-r_{eq})+\frac{1}{2!}\frac{d^2U}{dr^2}|_{r=r_{eq}}(r-r_{eq}^2)+\frac{1}{3!}\frac{d^3U}{dr^3}|_{r=r_{eq}}(r-r_{eq}^3)+\cdots\\
		U(r)&=U(r_{eq}) + \xcancel{\frac{dU}{dr}|_{r=r_{eq}}(r-r_{eq}})+\frac{1}{2!}\frac{d^2U}{dr^2}|_{r=r_{eq}}(r-r_{eq}^2)+\xcancel{\frac{1}{3!}\frac{d^3U}{dr^3}|_{r=r_{eq}}(r-r_{eq}^3)}+\cdots\\
	\end{align*}

	So that:

	$$U(r_{AB}) = \frac{1}{2}k_{AB}(r_{AB}-r_{AB,eq})^2$$


		\subsubsection{Anharmonic force constant}

		$$U(r_{AB}) = \frac{1}{2}[k_{AB}+k^{(3)}_{AB}(r_{AB}-r_{AB, eq})](r_{AB}-r_{AB, eq})^2$$

		\subsubsection{Quartic correction}

		$$U(r_{AB}) = \frac{1}{2}[k_{AB}+k^{(3)}_{AB}(r_{AB}-r_{AB, eq}) + k^{(4)}_{AB}(r_{AB}-r_{AB,eq})^2](r_{AB}-r_{AB, eq})^2$$

		\subsubsection{Morse potential}

		$$U(r_{AB}) = D_{AB}[1-e^{-\alpha_{AB}(r_AB-r_{AB,eq})^2}]$$

\section{Valence angle bending}

$$U(\theta_{ABC}) = \frac{1}{2}[k_{ABC}+k^{(3)}_{ABC}(\theta_{ABC}-\theta_{ABC,eq})+k^{(4)}_{ABC}(\theta_{ABC}-\theta_{ABC,eq})^2+\cdots](\theta_{ABC}-\theta_{ABC, eq})^2$$

\section{Torsions}

$$U(\omega_{ABCD}) = \frac{1}{2}\sum\limits_{\{j\}_{ABCD}}V_{j,ABCD}[1+(-1)^{j+1}\cos(j\omega_{ABCD}+\psi_{j,ABCD})]$$

	\subsection{Improper torsions}

$$U(\omega_{ABCD}) = \frac{1}{2}\sum\limits_{\{j\}_{ABCD}}V_{j,ABCD}[1+(-1)^{j+1}\cos(j\omega_{ABCD}+\psi_{j,ABCD})]$$

\section{Van der Waals interactions}
Some force fields reduce $1,4$-interactions by a scale factor through torsions.

	\subsection{Lennard-Jones potential}

	\begin{align*}
		U_(r_{AB}) &= \frac{a_{AB}}{r^{12}_{AB}}-\frac{b_{AB}}{r^6_{AB}}=\\
							 &= 4\epsilon_{AB}\biggl[\biggl(\frac{\sigma_{AB}}{r_{AB}}\biggr)^{12}-\biggl(\frac{\sigma_{AB}}{r_{AB}}\biggr)^6\biggr]
	\end{align*}

	And the distance with minimum energy:

	$$r^*_{AB} = 2^{\frac{1}{6}}\sigma_{AB}$$

	\subsection{Morse potential}

	$$U(r_{AB}) = D_{AB}[1-e^{-a_{AB}(r_{AB}-r_{AB,eq})^2}]$$

	\subsection{Hill potential}

	$$U(r_{AB}) = \epsilon_{AB}\biggl[\frac{6}{\beta_{AB}-6}e^{\beta_{AB}\frac{1-r_{AB}}{r^*_{AB}}}-\frac{\beta_{AB}}{\beta_{AB}-6}\biggl(\frac{r^*_{AB}}{r_{AB}}\biggr)^6\biggr]$$


\section{Electrostatic interactions}
Consider all the electrostatic interactions:

$$U_{AB} = \vec{M}^{(A)}V^{(B)}$$

Now, summing over all molecules:

$$U_{AB} = \sum\limits_{A}\sum\limits_{B>A}\vec{M}^{(A)}\vec{V}^{(B)}$$

	\subsection{Point like charges}
	All atoms are considered as point-like charges:

	$$U_{AB} = \frac{q_Aq_B}{\epsilon_{AB}r_{AB}}$$

	\subsection{Dipolar interactions}

	$$U_{AB/CD} = \frac{\mu_{AB}\mu_{CD}}{\epsilon_{AB/CD}r^3_{AB/CD}}(\cos\chi_{AB/CD}-3\cos\alpha_{AB}\cos\alpha_{CD})$$

	\subsection{Dielectric constants}

	$$U_{AB} = \frac{q_Aq_B}{\epsilon_{AB}r_{AB}}$$

	$$\epsilon_{AB} = \begin{cases}\infty&\text{ if }A\land B\text{are 1,2- or 1,3-related}\\3.0&\text{ if }A\land B\text{are 1,4-related}\\1.5&\text{otherwise}\end{cases}$$

	\subsection{Cross terms}

	\begin{align*}
		U(\vec{q}) = &U(\vec{q}_{eq}) + \sum\limits_{i=1}^{3N-6}(q_i-q_{i,eq})\frac{\partial U}{\partial q_i}|_{\vec{q}=\vec{q}_{eq}} + \\
								 &+\frac{1}{2!}\sum\limits_{i=1}^{3N-6}\sum\limits_{j=1}^{3N-6}(q_i-q_{i.eq})(q_j-q_{j,eq})\frac{\partial^2 U}{\partial q_i\partial q_j}|_{\vec{q}=\vec{q}_{eq}} +\\
								 &=\frac{1}{3!}\sum\limits_{i=1}^{3N-6}\sum\limits_{j=1}^{3N-6}\sum\limits_{k=1}^{3N-6}(q_i-q_{i,eq})(q_j-q_{j.eq})(q_k-q_{k,eq})\frac{\partial^3 U}{\partial q_i\partial q_j\partial q_k}|_{\vec{q}=\vec{q}_{eq}} + \cdots
	\end{align*}

	$$U(r_{AB}, \theta_{ABC}) = \frac{1}{2}k_{AB,ACB}(r_{AB}-r_{AB, eq})(\theta_{ABC}-\theta_{ABC, eq})$$

	\subsection{Parametrization}
	Let:

	$$Z = \biggl[\sum\limits_{i}^{observables}\sum\limits_{j}^{occurrences}\frac{(calc_{i,j}-expt_{i,j})^2}{w_i^2}\biggr]^{\frac{1}{2}}$$

	The penalty function.
	Consider the number of parameters:

	$$p = N + (N-1)+(N-2)+\cdots = N\frac{N+1}{2}$$

	A possible strategy for parametrization is:

	\begin{align*}
		\sigma_{AB} &= \sigma_A+\sigma_B
		\epsilon_{AB} &= (\epsilon_A\epsilon_B)^{\frac{1}{2}}
	\end{align*}

\section{Force field energies}

	\subsection{Geometry optimization}

	$$\vec{g}(\vec{q}) = \begin{bmatrix} \frac{\partial U}{\partial q_1} \\ \frac{\partial U}{\partial q_2} \\ \vdots \\ \frac{\partial U}{\partial q_n}\end{bmatrix}$$

	Such that the cost reaches a global minimum $J_{min}(\vec{w})$.

	\subsection{Derivative of the potential function}

	$$\frac{\partial U}{\partial x_A} = \sum\limits_{i\in A}\frac{\partial U}{\partial r_{Ai}}\frac{r_{Ai}}{\partial x_A}$$

	$$U(r_{AB}) = \frac{1}{2}[k_{AB}+k_{AB}^{(3)}(r_{AB}-r_{AB, eq}) + k_{AB}^{(4)}(r_{AB}-r_{AB, eq})^2](r_{AB}-r_{AB,eq})^2$$

	$$\frac{\partial U}{\partial r_{Ai}} = \frac{1}{2}[2k_{Ai}+3k_{Ai}^{(3)}r_{Ai}-r_{Ai, eq}) + 4k^{(4)}_{Ai}(r_{Ai}-r_{Ai, eq})^2](r_{Ai}-r_{Ai, eq})$$

	$$\frac{\partial r_{Ai}}{\partial x_A} = \frac{x_A-x_i}{\sqrt{(x_A-x_i)^2+(y_A-y_i)^2+(z_A-z_i)^2}}$$

		\subsubsection{Newton-Raphson}

		$$U(\vec{q}^{(k+1)}) = U(\vec{q}^{(k)}) + (\vec{q}^{(k+1)} -\vec{q}^{(k)})\vec{g}^{(k)} + \frac{1}{2}(\vec{q}^{(k+1)}-\vec{q}^{(k)})H^{(k)}(\vec{q}^{(k+1)}-\vec{q}^{(k)})$$

		Where $H$ is the Hessian matrix built:

		$$H_{ij}^{(k)} = \frac{\partial^2 U}{\partial q_i\partial q_j}|_{\vec{q}=\vec{q}^{(k)}}$$

		And:

		$$\frac{\partial U(\vec{q}^{(k+1)})}{\partial q_i^{(k+1)}} = \frac{\partial \vec{q}^{(k+1)}}{\partial q_i^{(k+1)}}\vec{g}^{(k)}+ \frac{1}{2}\frac{\partial \vec{q}^{(k+1)}}{\partial q_i^{(k+1)}}H^{(k)}(\vec{q}^{(k+1)}-\vec{q}^{(k)})+\frac{1}{2}(\vec{q}^{(k+1)}-\vec{q}^{(k)})H^{(k)}\frac{\partial \vec{q}^{(k+1)}}{\partial q_i^{(k+1)}}$$

		Where:

		$$\vec{g}_i^{(k+1)} = \vec{g}_i^{(k)} + [H^{(k)}(\vec{q}^{(k+1)}-\vec{q}^{(k)})]_i$$

		With the stationary condition:

		$$\vec{0} = \vec{g}^{(k)} + H^{(k)}(\vec{q}^{(k+1)}-\vec{q}^{(k)})\Rightarrow \vec{q}^{(k+1)} = \vec{q}^{(k)} - [H^{(k)}]^{-1}\vec{g}^{(k)}$$

	\subsection{Types of force fields}
	Force fields can be categorized as:

	\begin{multicols}{2}
		\begin{itemize}
			\item All atoms: one atom corresponds to one bead.
			\item More atoms: more atoms correspond to one bead.
			\item Corse grained: groups of atoms correspond to one bead.
			\item Polarizable force fields: point charges are variables.
		\end{itemize}
	\end{multicols}

	As a golden rule parameters from different force fields should never be mixed.
