\chapter{Semi-empirical force fields}

\section{Potential energy surface}
The potential energy surface or PES is the landscape of what values the potential energy of an atom can assume.
Different points can be recognized like:

\begin{multicols}{2}
	\begin{itemize}
		\item Saddle point.
		\item Local maximums.
		\item Local minimum.
	\end{itemize}
\end{multicols}

	\subsection{Bond stretching}
	Let $U(r)$ the function of the potential energy, now the PES can be determined by vibrational spectroscopy.
	Let $r_{eq}$ the distance for which the potential energy is minimal, then performing a Taylor expansion of the potential energy function:

	\begin{align*}
		U(r)&=U(r_{eq}) + \frac{dU}{dr}|_{r=r_{eq}}(r-r_{eq})+\frac{1}{2!}\frac{d^2U}{dr^2}|_{r=r_{eq}}(r-r_{eq}^2)+\frac{1}{3!}\frac{d^3U}{dr^3}|_{r=r_{eq}}(r-r_{eq}^3)+\cdots\\
		U(r)&=U(r_{eq}) + \xcancel{\frac{dU}{dr}|_{r=r_{eq}}(r-r_{eq}})+\frac{1}{2!}\frac{d^2U}{dr^2}|_{r=r_{eq}}(r-r_{eq}^2)+\xcancel{\frac{1}{3!}\frac{d^3U}{dr^3}|_{r=r_{eq}}(r-r_{eq}^3)}+\cdots\\
	\end{align*}

	So that:

	$$U(r_{AB}) = \frac{1}{2}k_{AB}(r_{AB}-r_{AB,eq})^2$$


		\subsubsection{Anharmonic force constant}

		$$U(r_{AB}) = \frac{1}{2}[k_{AB}+k^{(3)}_{AB}(r_{AB}-r_{AB, eq})](r_{AB}-r_{AB, eq})^2$$

		\subsubsection{Quartic correction}

		$$U(r_{AB}) = \frac{1}{2}[k_{AB}+k^{(3)}_{AB}(r_{AB}-r_{AB, eq}) + k^{(4)}_{AB}(r_{AB}-r_{AB,eq})^2](r_{AB}-r_{AB, eq})^2$$

		\subsubsection{Morse potential}

		$$U(r_{AB}) = D_{AB}[1-e^{-\alpha_{AB}(r_AB-r_{AB,eq})^2}]$$

\section{Valence angle bending}

$$U(\theta_{ABC}) = \frac{1}{2}[k_{ABC}+k^{(3)}_{ABC}(\theta_{ABC}-\theta_{ABC,eq})+k^{(4)}_{ABC}(\theta_{ABC}-\theta_{ABC,eq})^2+\cdots](\theta_{ABC}-\theta_{ABC, eq})^2$$

\section{Torsions}

$$U(\omega_{ABCD}) = \frac{1}{2}\sum\limits_{\{j\}_{ABCD}}V_{j,ABCD}[1+(-1)^{j+1}\cos(j\omega_{ABCD}+\psi_{j,ABCD})]$$

	\subsection{Improper torsions}

$$U(\omega_{ABCD}) = \frac{1}{2}\sum\limits_{\{j\}_{ABCD}}V_{j,ABCD}[1+(-1)^{j+1}\cos(j\omega_{ABCD}+\psi_{j,ABCD})]$$

\section{Van der Waals interactions}
Some force fields reduce $1,4$-interactions by a scale factor through torsions.

	\subsection{Lennard-Jones potential}

	\begin{align*}
		U_(r_{AB}) &= \frac{a_{AB}}{r^{12}_{AB}}-\frac{b_{AB}}{r^6_{AB}}=\\
							 &= 4\epsilon_{AB}\biggl[\biggl(\frac{\sigma_{AB}}{r_{AB}}\biggr)^{12}-\biggl(\frac{\sigma_{AB}}{r_{AB}}\biggr)^6\biggr]
	\end{align*}

	And the distance with minimum energy:

	$$r^*_{AB} = 2^{\frac{1}{6}}\sigma_{AB}$$

	\subsection{Morse potential}

	$$U(r_{AB}) = D_{AB}[1-e^{-a_{AB}(r_{AB}-r_{AB,eq})^2}]$$

	\subsection{Hill potential}

	$$U(r_{AB}) = \epsilon_{AB}\biggl[\frac{6}{\beta_{AB}-6}e^{\beta_{AB}\frac{1-r_{AB}}{r^*_{AB}}}-\frac{\beta_{AB}}{\beta_{AB}-6}\biggl(\frac{r^*_{AB}}{r_{AB}}\biggr)^6\biggr]$$


\section{Electrostatic interactions}
Consider all the electrostatic interactions:

$$U_{AB} = \vec{M}^{(A)}V^{(B)}$$

Now, summing over all molecules:

$$U_{AB} = \sum\limits_{A}\sum\limits_{B>A}\vec{M}^{(A)}\vec{V}^{(B)}$$

	\subsection{Point like charges}
	All atoms are considered as point-like charges:

	$$U_{AB} = \frac{q_Aq_B}{\epsilon_{AB}r_{AB}}$$

	\subsection{Dipolar interactions}

	$$U_{AB/CD} = \frac{\mu_{AB}\mu_{CD}}{\epsilon_{AB/CD}r^3_{AB/CD}}(\cos\chi_{AB/CD}-3\cos\alpha_{AB}\cos\alpha_{CD})$$

	\subsection{Dielectric constants}

	$$U_{AB} = \frac{q_Aq_B}{\epsilon_{AB}r_{AB}}$$

	$$\epsilon_{AB} = \begin{cases}\infty&\text{ if }A\land B\text{are 1,2- or 1,3-related}\\3.0&\text{ if }A\land B\text{are 1,4-related}\\1.5&\text{otherwise}\end{cases}$$

	\subsection{Cross terms}

	\subsection{Parametrization}

\section{Force field energies}

	\subsection{Geometry optimization}

	\subsection{Taking derivatives}

		\subsubsection{Newton-Raphson}

	\subsection{Types of force fields}
