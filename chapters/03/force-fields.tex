\chapter{Semi-empirical force fields}

\section{Potential energy surface}
The potential energy surface or PES is the landscape of what values the potential energy of an atom can assume.
Different points can be recognized like:

\begin{multicols}{2}
	\begin{itemize}
		\item Saddle point.
		\item Local maximums.
		\item Local minimum.
	\end{itemize}
\end{multicols}

	\subsection{Bond stretching}
	Let $U(r)$ the function of the potential energy, now the PES can be determined by vibrational spectroscopy.
	Let $r_{eq}$ the distance for which the potential energy is minimal, then performing a Taylor expansion of the potential energy function:

	\begin{align*}
		U(r)&=U(r_{eq}) + \frac{dU}{dr}|_{r=r_{eq}}(r-r_{eq})+\frac{1}{2!}\frac{d^2U}{dr^2}|_{r=r_{eq}}(r-r_{eq}^2)+\frac{1}{3!}\frac{d^3U}{dr^3}|_{r=r_{eq}}(r-r_{eq}^3)+\cdots\\
		U(r)&=U(r_{eq}) + \xcancel{\frac{dU}{dr}|_{r=r_{eq}}(r-r_{eq}})+\frac{1}{2!}\frac{d^2U}{dr^2}|_{r=r_{eq}}(r-r_{eq}^2)+\xcancel{\frac{1}{3!}\frac{d^3U}{dr^3}|_{r=r_{eq}}(r-r_{eq}^3)}+\cdots\\
	\end{align*}

	So that:

	$$U(r_{AB}) = \frac{1}{2}k_{AB}(r_{AB}-r_{AB,eq})^2$$


		\subsubsection{Anharmonic force constant}

		$$U(r_{AB}) = \frac{1}{2}[k_{AB}+k^{(3)}_{AB}(r_{AB}-r_{AB, eq})](r_{AB}-r_{AB, eq})^2$$

		\subsubsection{Quartic correction}

		$$U(r_{AB}) = \frac{1}{2}[k_{AB}+k^{(3)}_{AB}(r_{AB}-r_{AB, eq}) + k^{(4)}_{AB}(r_{AB}-r_{AB,eq})^2](r_{AB}-r_{AB, eq})^2$$

		\subsubsection{Morse potential}

		$$U(r_{AB}) = D_{AB}[1-e^{-\alpha_{AB}(r_AB-r_{AB,eq})^2}]$$

\section{Valence angle bending}

$$U(\theta_{ABC}) = \frac{1}{2}[k_{ABC}+k^{(3)}_{ABC}(\theta_{ABC}-\theta_{ABC,eq})+k^{(4)}_{ABC}(\theta_{ABC}-\theta_{ABC,eq})^2+\cdots](\theta_{ABC}-\theta_{ABC, eq})^2$$

\section{Torsions}

	\subsection{Improper torsions}

\section{Van der Waals interactions}

	\subsection{Other potentials fro Van der Waals interactions}

\section{Electrostatic interactions}

	\subsection{Point like charges}

	\subsection{Dipolar interactions}

	\subsection{Dielectric constants}

	\subsection{Cross terms}

	\subsection{Parametrization}

\section{Force field energies}

	\subsection{Geometry optimization}

	\subsection{Taking derivatives}

		\subsubsection{Newton-Raphson}

	\subsection{Types of force fields}
